%!TeX program = XeLaTeX
\documentclass[10pt,mathserif]{beamer}
%,aspectratio=169 %画面比例16:9
\usepackage[English]{XDstyle}
%\usepackage[English,CircleLogo,XDblue]{XDstyle}


\title{Title}
\subtitle{Subtitle}
\institute{School of Mathematics and Statistics,~Xidian University\\Email:~$\href{mailto:Stick_Cui@163.com}{{Stick\_Cui@163.com}}$}
\author{Stick Cui}
\date{\today}
	
\begin{document}
{\mybg \frame[plain,noframenumbering]{\titlepage}}
%\part{第一部分}
\section{Section One}
\begin{frame}{Usual Test}
There is a book on the desk.
\end{frame}

\section{Theorem Test}
\frame{\frametitle{Theorem Environment}
\begin{theorem}
This is a theorem
\end{theorem}

\begin{lemma}
This is a lemma.
\end{lemma}

\begin{definition}
This is a definition.
\end{definition}
}

\section{Reference}
\frame{
\frametitle{Reference}
Journal\cite{RefJ},~Journal\cite{RefB}.
\begin{thebibliography}{}
\bibitem{RefJ}
	Fu, JL.,\& Chen, QX. (2005). Study on topic partition based on sequential paragraphic similarity. \emph{Computer Applications}. 25(9), 2022-2024.

\bibitem{RefB}
	Qiao, HG. Based on The Improved Method of Literature Evaluation Research of PageRank Algorithm[DB/OL]. http://www.docin.com/p-779713218.html?qq-pf-to=pcqq.group. 2015/1/24.
\end{thebibliography}
}

\section{Figure Test}
\frame{\frametitle{Figure Test}
\begin{figure}[!h]
\centering
\includegraphics[width=0.5\textwidth]{./Image/figure1}
\caption{This is just a figure for testing.}\label{Fig.figure1}
\end{figure}
}

\frame{
\frametitle{Equation}
\[\sigma = \sin{\theta}\]
\begin{equation}
\rho = \sum_{i=0}^n{a_i}
\end{equation}
}

{\mybg
\begin{frame}[plain,noframenumbering]
 \finalpage{\yihao Thank You!}
\end{frame}}

\end{document}